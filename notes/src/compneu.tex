% Options for packages loaded elsewhere
\PassOptionsToPackage{unicode}{hyperref}
\PassOptionsToPackage{hyphens}{url}
\PassOptionsToPackage{dvipsnames,svgnames*,x11names*}{xcolor}
%
\documentclass[
  12pt,
]{memoir}
\usepackage{lmodern}
\usepackage{amssymb,amsmath}
\usepackage{ifxetex,ifluatex}
\ifnum 0\ifxetex 1\fi\ifluatex 1\fi=0 % if pdftex
  \usepackage[T1]{fontenc}
  \usepackage[utf8]{inputenc}
  \usepackage{textcomp} % provide euro and other symbols
\else % if luatex or xetex
  \usepackage{unicode-math}
  \defaultfontfeatures{Scale=MatchLowercase}
  \defaultfontfeatures[\rmfamily]{Ligatures=TeX,Scale=1}
\fi
% Use upquote if available, for straight quotes in verbatim environments
\IfFileExists{upquote.sty}{\usepackage{upquote}}{}
\IfFileExists{microtype.sty}{% use microtype if available
  \usepackage[]{microtype}
  \UseMicrotypeSet[protrusion]{basicmath} % disable protrusion for tt fonts
}{}
\makeatletter
\@ifundefined{KOMAClassName}{% if non-KOMA class
  \IfFileExists{parskip.sty}{%
    \usepackage{parskip}
  }{% else
    \setlength{\parindent}{0pt}
    \setlength{\parskip}{6pt plus 2pt minus 1pt}}
}{% if KOMA class
  \KOMAoptions{parskip=half}}
\makeatother
\usepackage[table]{xcolor}

\IfFileExists{xurl.sty}{\usepackage{xurl}}{} % add URL line breaks if available
\IfFileExists{bookmark.sty}{\usepackage{bookmark}}{\usepackage{hyperref}}
\hypersetup{
  pdftitle={Computational Neuroscience},
  pdfauthor={Susmit Islam, MBBS},
  colorlinks=true,
  linkcolor=blue,
  filecolor=Maroon,
  citecolor=Blue,
  urlcolor=blue,
  pdfcreator={LaTeX via pandoc}}
\urlstyle{same} % disable monospaced font for URLs
\usepackage[margin=2cm]{geometry}
\usepackage{graphicx}
\makeatletter
\def\maxwidth{\ifdim\Gin@nat@width>\linewidth\linewidth\else\Gin@nat@width\fi}
\def\maxheight{\ifdim\Gin@nat@height>\textheight\textheight\else\Gin@nat@height\fi}
\makeatother
% Scale images if necessary, so that they will not overflow the page
% margins by default, and it is still possible to overwrite the defaults
% using explicit options in \includegraphics[width, height, ...]{}
\setkeys{Gin}{width=\maxwidth,height=\maxheight,keepaspectratio}
% Set default figure placement to htbp
\makeatletter
\def\fps@figure{htbp}
\makeatother
\setlength{\emergencystretch}{3em} % prevent overfull lines
\providecommand{\tightlist}{%
  \setlength{\itemsep}{0pt}\setlength{\parskip}{0pt}}
\setcounter{secnumdepth}{-\maxdimen} % remove section numbering
\usepackage{graphicx}
\usepackage{subcaption}
\usepackage{hyperref}
\usepackage{pdfpages}
\usepackage{palatino}
\usepackage{tikz}
\usepackage{pgfplots}
\usetikzlibrary{positioning,arrows.meta}

\title{Computational Neuroscience}
\author{Susmit Islam, MBBS}
\date{}

\begin{document}
\frontmatter
\includepdf[pages=1,fitpaper]{../pdfs/cover-compneu.pdf}
\maketitle

\mainmatter
\renewcommand{\labelitemi}{$\blacktriangleright$}
\renewcommand{\labelitemii}{$\boldsymbol\circ$}
\renewcommand{\labelitemiii}{$\bullet$}
\renewcommand{\labelitemiv}{\tiny$\blacksquare$}

\openany
\raggedbottom
\twocoltocetc
\tableofcontents

\pagebreak

\hypertarget{preface}{%
\chapter*{Preface}\label{preface}}
\addcontentsline{toc}{chapter}{Preface}

What follows are my notes on computational neuroscience, acquired mostly
through self-study. The major sources that have been consulted are as
follows.

\begin{itemize}
    \tightlist
    \item \href{https://www.coursera.org/learn/computational-neuroscience}{Computational Neuroscience}, an online course on Coursera by the University of Washington.
    \item \href{https://neuronaldynamics.epfl.ch/online/}{Neuronal Dynamics}, a freely available online textbook.
    \item Lecture notes on Computational Neuroscience by Todd Troyer.
\end{itemize}

As these are the fruits of quite unguided explorations through the vast
terrains of computational neuroscience, it's astronomically unlikely
that these are void of errors. I take full responsibility for all
mistakes within. But in exchange, I urge you, the reader, to do me a
couple of favours. First, read everything (these notes \emph{and}
everything else that you will ever read) with a hint of healthy
scepticism - all the authors are, just like you and I, humans. And
second, let me know of the errors you find here - be they ``hard
errors'' (i.e.~something scientifically incorrect), or ``soft errors''
(i.e., something that isn't, strictly speaking, incorrect, but could be
presented better). Computational biology is a joy of a ride - I hope you
have as much fun reading these notes as I have had in preparing these.

\hfill Susmit Islam

\hfill 2022-09-21

\hypertarget{intro}{%
\chapter{Intro}\label{intro}}

\hypertarget{types-of-models}{%
\section{Types of Models}\label{types-of-models}}

\begin{itemize}
\tightlist
\item
  Descriptive \(\rightarrow\) \emph{what?}
\item
  Mechanistic \(\rightarrow\) \emph{how?}
\item
  Interpretive \(\rightarrow\) \emph{why?}
\end{itemize}

\hypertarget{receptive-fields}{%
\section{Receptive fields}\label{receptive-fields}}

\begin{itemize}
\tightlist
\item
  Patterns of activation of a receptor
\item
  Retinal ganglion cells have a centre-surround pattern of activation
\item
  Cells in the V1 or V2 cortex have more complex patterns of activation
  e.g. oriented bars or more complex features
\item
  Each layer builds on top of the previous ones to represent more and
  more complex patterns
\item
  Our task:

  \begin{itemize}
  \tightlist
  \item
    \emph{Describe} the receptive fields at each layer (\emph{what?})
  \item
    Explain the \emph{mechanism} by which the higher-order receptive
    fields arise out of the lower-order fields (\emph{how})
  \item
    \emph{Interpret} the computational advantages and disadvantages of
    the obtained models over other possible models (\emph{why})
  \end{itemize}
\end{itemize}

\backmatter
\end{document}
